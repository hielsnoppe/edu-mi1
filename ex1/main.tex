\documentclass[11pt,a4paper]{article}
\usepackage{amsmath}
\usepackage[a4paper, top=1in, bottom=1.25in, left=0.75in, right=0.75in]{geometry}

\begin{document}

\section{Distributions and expected values}

\begin{enumerate}

\item[a)]

Requirement for $p(x)$ to be probability density function (PDF):

\begin{displaymath}
\int_{-\infty}^{\infty} p(x) \, dx = 1
\end{displaymath}

Calculate antiderivative $P(x)$:

\begin{displaymath}
P(x) = \begin{cases}
-c \cdot cos(x) + const.   & x \in [0, \pi]\\
const.                      & else
\end{cases}
\end{displaymath}

Choose $const.$ to be $0$.

\begin{eqnarray*}
1   &   = & \int_{-\infty}^\infty p(x) \, dx\\
    &   = & \int_0^\pi p(x) \, dx\\
    &   = & \Big[ P(x) \Big]_0^\pi\\
    &   = & -c \cdot cos(\pi) + c \cdot cos(0)\\
    &   = & 2c\\
0.5 &   = & c
\end{eqnarray*}

\item[b)]

Expected value of PDF $p(x)$:

\begin{displaymath}
\left\langle X \right\rangle_p = \int_{-\infty}^\infty x \cdot p(x) \, dx
\end{displaymath}

Calculate antiderivative $Q(x)$ of $x \cdot p(x)$:

\begin{displaymath}
Q(x) = \begin{cases}
0.5 \, (sin(x) - x \cdot cos(x))    & x \in [0, \pi]\\
const.                              & else
\end{cases}
\end{displaymath}

Choose $const.$ to be $0$.

\begin{eqnarray*}
\left\langle X \right\rangle_p  &   = & \int_{-\infty}^\infty x \cdot p(x) \, dx\\
                                &   = & \int_0^\pi x \cdot p(x) \, dx\\
                                &   = & \Big[ Q(x) \Big]_0^\pi\\
                                &   = & 0.5 \, (sin(\pi) - \pi \cdot cos(\pi)) - 0.5 \, (sin(0) - 0 \cdot cos(0))\\
                                &   = & \frac{\pi}{2}
\end{eqnarray*}

\item[c)]

\end{enumerate}

\section{Marginal densities}

\section{Taylor expansion}

General form of the taylor series:

\begin{displaymath}
\sum_{n=0}^\infty \frac{f^{(n)}(x_0)}{n!} (x - x_0)^n
\end{displaymath}

Calculate derivatives up to $n=3$:

\begin{eqnarray*}
f'(x)   &   = & \frac{1}{2 \, \sqrt{x + 1}}\\
f''(x)  &   = & - \frac{1}{4 \, (x + 1)^\frac{3}{2}}\\
f'''(x) &   = & \frac{3}{8 \, (x + 1)^\frac{5}{2}}\\
        &   = & 1 + \frac{x}{2} + \frac{x^2}{8} + \frac{x^3}{16} + \ldots
\end{eqnarray*}

\section{Determinant of a matrix}

\begin{eqnarray*}
A   &   = & \begin{pmatrix}
  5 &   8 &  16\\
  4 &   1 &   8\\
 -4 &  -4 & -11
\end{pmatrix}\\
det(A)  &   = & a_{11} a_{22} a_{33} + a_{12} a_{23} + a_{31} + a_{13} a_{21} a_{32}
            - a_{31} a_{22} a_{13} - a_{32} a_{23} a_{11} - a_{33} a_{21} a_{12}\\
        &   = & 5 \cdot 1 \cdot (-11) + 8 \cdot 8 \cdot (-4) + 16 \cdot 4 \cdot (-4)
            - (-4) \cdot 1 \cdot 16 - (-4) \cdot 8 \cdot 5 - (-11) \cdot 4 \cdot 8\\
        &   = & -55 - 256 - 256 + 64 + 160 + 352\\
        &   = & 9\\
tr(A)   &   = & a_{11} + a_{22} + a_{33}\\
        &   = & 5 + 1 - 11\\
        &   = & -5
\end{eqnarray*}

\section{Critical points}

\section{Bayes rule}

\section{Learning paradigms}

\end{document}
