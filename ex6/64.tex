\documentclass[11pt,a4paper]{article}
\usepackage{amsmath}
\usepackage{enumerate}
\usepackage{fancyhdr}
\usepackage[utf8]{inputenc}
\usepackage[a4paper, top=1in, bottom=1.25in, left=0.75in, right=0.75in]{geometry}

\begin{document}
\pagestyle{fancy}
\fancyhead[R]{Robert Schüle 319782, Christoph Ende 331655}

%%%%%%%%%%%%%%%%%%%%%%%%%%%%%%%%%%%%%%%%%%%%%%%%%%%%%%%%%%%%%%%%%%%%%%%%%%%%%%%%
\subsection*{6.3 c)}

Since the variances of the input data distribution in a) is much higher and there is an overlap between the classes, the plot wouldn't appear to be so structured, which means
that there would be wrong class assignments especially around the point where $w^T\,\phi(x)$ becomes zero. If a linear classifier would be applied, the data structure
would be more difficult to approximate which would result in even more false classifications.

\subsection*{6.4 Kullback-Leibler Distance}

\begin{eqnarray}
D_{KL}(p||q)\;=\; \sum_{x \in \chi} p(x)\,log \frac{p(x)}{q(x)} &\;=\;& \mathbf{E}_p \Big{[} log \frac{p(X)}{q(X)} \Big{]} \;\;\;\;\; \Big{|} \mbox{X is a random variable} \\
&\;=\;& \mathbf{E}_p \Big{[} -log \frac{q(X)}{p(X)} \Big{]} \;\;\;\;\; \Big{|} \mbox{use Jensens Inequality}  \\
&\;\geq \;& -log\; \mathbf{E}_p \Big{[} \frac{q(X)}{p(X)} \Big{]} \\
&\;=\;& -log \sum_{x \in \chi}\;p(x)\, \frac{q(x)}{p(x)} \\
&\;=\;& -log \sum_{x \in \chi}\; q(x) \\
&\;=\;& -log\,1 \;=\; 0
\end{eqnarray}



\end{document}
